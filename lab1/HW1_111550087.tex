\documentclass[a4paper, 12pt]{article}
\usepackage{fontspec}
\usepackage{xeCJK}
\usepackage{mathtools}
\usepackage{tocloft}
\usepackage{graphicx}
\usepackage{indentfirst}
\usepackage{float}
\setCJKmainfont{微軟正黑體}
\XeTeXlinebreaklocale "zh"
\XeTeXlinebreakskip = 0pt plus 1pt
\begin{document}
\title{HW1\_Report}
\author{林炫廷 111550087}
\date{\today}
\maketitle
{\bf1. What is the usage of \$zero? What happen if you execute?}\par
\begin{center}
    \begin{tabular}{|c|}
        \hline
        addi \$zero, \$zero, 5 \\
        \hline
    \end{tabular}
\end{center}
{\bf Ans: }\\
{\it First.}\par
1. This register could be a constant zero used in arithmetic and
logical operation. Also, it is admitted to be considered as value zero in the instructions.\par
2. It could be used for register initialization, making it easy to clear the variables and registers.\par
3. It is commonly used in comparisons and branching. For example: {\sl bne} and {\sl beq}.\\
{\it Second.}\par
The register \$zero reserves the value zero. Unlike other general-purpose registers,
you cannot modify the value of \$zero directly.
Any attempt to write to \$zero is ignored, and it remains set to zero.
As a result, this instruction does not affect register \$zero. and it retains its value of zero.
\newpage
{\bf 2. How to use the stack to ensure that the value of each register is correctly saved when executing a recursive function? }\\
{\bf Ans: }\par
After entering a function, we simply use stack to store the return address and other parameters.
This allows us to obtain the correct parameters
within the function layer and return to the correct line of code.
The table below illustrates this process:
\begin{center}
    \begin{tabular}{|c|}
        \hline
        addi	\$sp, \$sp, -8 \\
        \hline
        sw		\$ra, 4(\$sp)   \\
        \hline
        sw		\$a0, 0(\$sp)   \\
        \hline
    \end{tabular}
\end{center}\par
By utilizing the stack,
we maintain a hierarchical structure of function calls,
ensuring the integrity of parameter passing and return
addresses throughout the program execution.\\\par
{\bf 3. What was the most challenging part for you in this homework? }\\
{\bf Ans: }\par
There are lots of obstacles that I have encountered while working on this homework assignment.
The main chellenge I face is that MIPS code is quite abstract, making it diffcult to determine
whether the program is prograssing correctly.
To implement the for loop function in MIPS, I have searched through the class slide
and gathering information from Internet.\par
It is obvious that debugging
the code is more challenging compared to highed-level languages.
This is partically true when doing recursive functions.
Despite carefully reading each line of code, I still struggle to identify why
the program's output is consistently incorrect.
Thus, I have resorted to closely observing the change in registers after executing each instruction.
By employing this approach, I have been able to identify bugs wothin the complex program structure.\par
Although it is not easy to solve these qusetion by using the unfamiliar language,
It has signigicantly accelerated my famility with assembly language.
I believe this homework is truly valuable in enhancing my understanding of this course.
\end{document}
\documentclass[a4paper, 12pt]{article}
\usepackage{fontspec}
\usepackage{xeCJK}
\usepackage{mathtools}
\usepackage{tocloft}
\usepackage{graphicx}
\usepackage{indentfirst}
\usepackage{float}
\setCJKmainfont{微軟正黑體}
\XeTeXlinebreaklocale "zh"
\XeTeXlinebreakskip = 0pt plus 1pt
\begin{document}
\title{HW3\_Report}
\author{林炫廷 111550087}
\date{\today}
\maketitle
{\bf1. Which part of the module in simple-single-CPU is redundant? Can you design a new instruction to use this module?}\\
{\bf Ans: }\\
Zero\_Filled. We can design the following instruction to use this module.
We take the data in register "rs" as the multiplicand, "rt" as the product, the 16-bit constant as the multiplier.
To carry out this instruction using Booth's algorithm or traditional method, we need to use the "zero\_filled" module
to extend 16-bit constant to 32-bit constant.
\begin{center}
    \begin{tabular}{|c|}
        \hline
        muli	\$rs, \$rt, 16\_bit constant \\
        \hline
    \end{tabular}
\end{center}\par
{\bf 2. Which instruction is redundant and why? }\\
{\bf Ans: }\par
bnez. We can achieve the same result by using the "bne" instruction and setting the register "rt" to "\$zero".
This change allows us to obtain the identical solution.
\end{document}